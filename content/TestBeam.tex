\chapter{Test beam at the CERN Proton Synchrotron}
To verify the performance of the new readout scheme, an IROC prototype from the pre-production phase equipped with the new front-end cards was taken to the Proton Synchrotron (PS) at CERN. The PS accelerates protons and collides them with a fixed target. In these collisions mainly pions and electrons are created, which could then be detected by the IROC.
\section{Testing the new readout scheme}
The main purpose of the test was to check the new readout electronics that have been developed to cope with the continuous readout. The front-end cards were completely redesigned and now contain the new SAMPA chips. However, only six front-end card were available at the time of the test beam so only about 20\% of the pad plane was equipped. 
\section{Experimental setup}
\section{Data aquisition}
\section{Track reconstruction}